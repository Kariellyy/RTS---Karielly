% % ----------------------------------------------------------
% % Modelagem do Projeto
% % ----------------------------------------------------------
% \chapter{Modelagem do Projeto}

% Esta seção deve detalhar como o projeto foi planejado e modelado, incluindo a estrutura e os processos que serão implementados.

% As notas de rodapé são detalhadas pela NBR 14724:2011 na seção 5.2.1\footnote{As
% notas devem ser digitadas ou datilografadas dentro das margens, ficando
% separadas do texto por um espaço simples de entre as linhas e por filete de 5
% cm, a partir da margem esquerda. Devem ser alinhadas, a partir da segunda linha
% da mesma nota, abaixo da primeira letra da primeira palavra, de forma a destacar
% o expoente, sem espaço entre elas e com fonte menor.}

% \section{Levantamento de Requisitos}
% Descreva o processo de coleta e análise dos requisitos do sistema. Explique como foram identificadas as necessidades dos usuários e as funcionalidades que o software deve oferecer. Inclua requisitos funcionais (o que o sistema deve fazer) e não funcionais (restrições e critérios de qualidade).



% \section{ Diagramas de Casos de Uso (opcional)}
% Se aplicável, apresente diagramas de casos de uso para ilustrar as interações entre os usuários e o sistema. Explique cada caso de uso e como ele contribui para o funcionamento geral do software.


%   \section{Diagramas de Classe (opcional)}
% Caso utilizado, descreva os diagramas de classe que representam a estrutura do sistema, incluindo as classes, seus atributos, métodos e relacionamentos. Explique como essas classes se organizam para atender aos requisitos do projeto.

% \section{Arquitetura do Sistema}
% Descreva a arquitetura geral do sistema, incluindo os componentes principais, suas interações e o fluxo de dados. Pode ser útil incluir diagramas de arquitetura, como MVC (Model-View-Controller) ou microsserviços.

% \section{Diagrama de Entidades-Relacionamentos (opcional)}
% Se o projeto envolve um banco de dados, apresente o diagrama de entidades-relacionamentos (DER) que modela as tabelas, seus atributos e os relacionamentos entre elas. Explique como o banco de dados foi projetado para atender às necessidades do sistema.

% \section{Interface}

% Descreva o design da interface do usuário (UI), incluindo aspectos como usabilidade, acessibilidade e experiência do usuário (UX). Se possível, inclua protótipos ou mockups das telas e explique como a interface foi planejada para ser intuitiva e eficiente.

% Exemplo de tabela:

% \index{tabelas}A \autoref{tab-nivinv} é um exemplo de tabela construída em
% \LaTeX.

% \begin{table}[htb]
% \ABNTEXfontereduzida
% \caption[Níveis de investigação]{Níveis de investigação.}
% \label{tab-nivinv}
% \begin{tabular}{p{2.6cm}|p{6.0cm}|p{2.25cm}|p{3.40cm}}
%   %\hline
%    \textbf{Nível de Investigação} & \textbf{Insumos}  & \textbf{Sistemas de Investigação}  & \textbf{Produtos}  \\
%     \hline
%     Meta-nível & Filosofia\index{filosofia} da Ciência  & Epistemologia &
%     Paradigma  \\
%     \hline
%     Nível do objeto & Paradigmas do metanível e evidências do nível inferior &
%     Ciência  & Teorias e modelos \\
%     \hline
%     Nível inferior & Modelos e métodos do nível do objeto e problemas do nível inferior & Prática & Solução de problemas  \\
%    % \hline
% \end{tabular}
% \legend{Fonte: \textcite{van86}}
% \end{table}

% Já a \autoref{tabela-ibge} apresenta uma tabela criada conforme o padrão do
% \textcite{macedo2005} requerido pelas normas da ABNT para documentos técnicos e
% acadêmicos.

% \begin{table}[htb]
% \IBGEtab{%
%   \caption{Um Exemplo de tabela alinhada que pode ser longa
%   ou curta, conforme padrão IBGE.}%
%   \label{tabela-ibge}
% }{%
%   \begin{tabular}{ccc}
%   \toprule
%    Nome & Nascimento & Documento \\
%   \midrule \midrule
%    Maria da Silva & 11/11/1111 & 111.111.111-11 \\
%   \midrule 
%    João Souza & 11/11/2111 & 211.111.111-11 \\
%   \midrule 
%    Laura Vicuña & 05/04/1891 & 3111.111.111-11 \\
%   \bottomrule
% \end{tabular}%
% }{%
%   \fonte{Produzido pelos autores.}%
%   \nota{Esta é uma nota, que diz que os dados são baseados na
%   regressão linear.}%
%   \nota[Anotações]{Uma anotação adicional, que pode ser seguida de várias
%   outras.}%
%   }
%   \end{table}
