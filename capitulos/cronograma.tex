\chapter{Cronograma}

O cronograma apresentado a seguir descreve a previsão de execução das atividades relacionadas ao desenvolvimento do sistema de registro de ponto digital. As etapas foram organizadas considerando o tempo disponível entre os meses de abril e agosto, e foram definidas de acordo com a complexidade de cada fase do projeto. O planejamento contempla desde o levantamento de requisitos até a finalização do protótipo e a redação do relatório técnico. 


\begin{table}[h]
    \centering
    \caption{Cronograma de atividades.}
    \renewcommand{\arraystretch}{1.3}
    \begin{tabular}{|l|c|c|c|c|c|}
        \hline
        \textbf{Atividades} & \textbf{Abr} & \textbf{Mai} & \textbf{Jun} & \textbf{Jul} & \textbf{Ago} \\
        \hline
        Levantamento de requisitos & X &  &   &   &   \\
        \hline
        Análise de soluções existentes e tecnologias &  X &  & &   &   \\
        \hline
        Definição da arquitetura do sistema &   X & X &  & &   \\
        \hline
        Desenvolvimento do protótipo &   &   X & X & X & \\
        \hline
        Testes de usabilidade &   &   &   &   X &  \\
        \hline
        Redação e ajustes do relatório técnico &   &   &  & X & X \\
        \hline
        Revisão final e entrega &   &   &   &   &  X \\
        \hline
    \end{tabular}
    \vspace{5pt}
    \caption*{Fonte: Elaborado pela autora.}
    \label{tab:cronograma}
\end{table}
