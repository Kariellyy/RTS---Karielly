% ---------------------------------------------------------- % Tecnologias Envolvidas % ---------------------------------------------------------- 
\chapter{Tecnologias Envolvidas} \label{cha:tecnologias}

Este capítulo apresenta as principais tecnologias que serão utilizadas no desenvolvimento do projeto, abrangendo tanto o front-end quanto o back-end. Serão descritas as linguagens de programação, frameworks, bibliotecas e ferramentas empregadas, bem como as razões para sua escolha e como cada uma contribui para o desenvolvimento do projeto.

\section{Tecnologias Front-End}

\subsection{Next.js}

O Next.js é um framework para aplicações web construído sobre o React, que permite a renderização no lado do servidor (Server-Side Rendering — SSR) e a geração de sites estáticos. Foi escolhido por oferecer, junto ao React, uma plataforma robusta, organizada e escalável, o que contribui diretamente para a qualidade e estrutura do desenvolvimento do projeto \cite{nextjs}.

\subsection{React.js}

O React.js é uma biblioteca JavaScript para criação de interfaces de usuário, baseada em componentes reutilizáveis. Foi escolhida por ser compatível com JavaScript e TypeScript, facilitar a modularização. \cite{reactjs}.

\subsection{TypeScript}

O TypeScript é uma linguagem de programação que adiciona tipagem estática ao JavaScript, permitindo identificar erros ainda durante o desenvolvimento. Foi escolhido por melhorar a legibilidade, a manutenção do código e por oferecer maior segurança no desenvolvimento \cite{typescript}.

\subsection{Tailwind CSS}

O Tailwind CSS é um framework de folhas de estilo em cascata (Cascading Style Sheets — CSS) baseado em classes utilitárias. Ele permite a criação rápida de interfaces responsivas e customizáveis, com menos necessidade de escrever CSS manualmente. Foi escolhido por agilizar o desenvolvimento visual e garantir consistência no design \cite{tailwindcss}.

\section{Tecnologias Back-End}

\subsection{NestJS}

O NestJS é um framework para construção de aplicações Node.js escaláveis e eficientes. Baseado em TypeScript, ele utiliza conceitos do Angular, como módulos, controladores e serviços, para estruturar o código de forma organizada e modular. O NestJS facilita a criação de APIs (Interfaces de Programação de Aplicações) robustas e de fácil manutenção \cite{nestjs}.

\subsection{Prisma}

O Prisma é um ORM (Mapeador Objeto-Relacional) moderno para Node.js e TypeScript. Ele simplifica a interação com o banco de dados, oferecendo uma API intuitiva e tipada para consultas e manipulação de dados. O Prisma facilita a manutenção da consistência dos dados e melhora a produtividade no desenvolvimento \cite{prisma}.

\subsection{PostgreSQL}

O PostgreSQL é um sistema de gerenciamento de banco de dados relacional de código aberto, conhecido por sua robustez, desempenho e conformidade com padrões. Ele suporta uma ampla variedade de tipos de dados e funcionalidades avançadas, sendo uma escolha sólida para aplicações que requerem integridade e escalabilidade dos dados \cite{postgresql}.

\section{Resumo das Tecnologias e Versões}

A Tabela \ref{tab:tecnologias-versoes} apresenta um resumo das principais tecnologias utilizadas no projeto, incluindo suas respectivas versões.

\begin{table}[!htbp]
\centering
\small
\caption{Resumo das tecnologias e versões utilizadas no projeto}
\label{tab:tecnologias-versoes}
\begin{tabular}{|p{0.3\textwidth}|p{0.2\textwidth}|p{0.2\textwidth}|p{0.2\textwidth}|}
\hline
\textbf{Tecnologia} & \textbf{Versão} & \textbf{Categoria} & \textbf{Propósito} \\
\hline
\multicolumn{4}{|c|}{\textbf{Frontend}} \\
\hline
Next.js & 15.3.5 & Framework & Renderização e roteamento \\
\hline
React & 19.0.0 & Biblioteca & Interface de usuário \\
\hline
TypeScript & 5.x & Linguagem & Tipagem estática \\
\hline
Tailwind CSS & 4.x & Framework CSS & Estilização responsiva \\
\hline
Radix UI & 1.x/2.x & Biblioteca & Componentes acessíveis \\
\hline
React Hook Form & 7.60.0 & Biblioteca & Gerenciamento de formulários \\
\hline
Zod & 3.25.75 & Biblioteca & Validação de dados \\
\hline
Recharts & 2.15.4 & Biblioteca & Gráficos e visualizações \\
\hline
\multicolumn{4}{|c|}{\textbf{Backend}} \\
\hline
NestJS & 11.0.1 & Framework & API e estrutura do servidor \\
\hline
TypeORM & 0.3.25 & ORM & Mapeamento objeto-relacional \\
\hline
PostgreSQL & 8.16.3 & Banco de Dados & Armazenamento de dados \\
\hline
JWT & 11.0.0 & Biblioteca & Autenticação e autorização \\
\hline
Passport & 0.7.0 & Biblioteca & Estratégias de autenticação \\
\hline
Bcrypt & 6.0.0 & Biblioteca & Criptografia de senhas \\
\hline
Class Validator & 0.14.2 & Biblioteca & Validação de dados \\
\hline
\multicolumn{4}{|c|}{\textbf{Ferramentas de Desenvolvimento}} \\
\hline
ESLint & 9.x & Linter & Análise estática de código \\
\hline
Prettier & 3.4.2 & Formatador & Formatação de código \\
\hline
Jest & 29.7.0 & Framework & Testes automatizados \\
\hline
Docker & - & Containerização & Deploy e isolamento \\
\hline
\end{tabular}
\end{table}

