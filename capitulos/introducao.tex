% ----------------------------------------------------------
% Introdução
% ----------------------------------------------------------
\chapter{Introdução}

A gestão da jornada de trabalho é um elemento crucial para o bom funcionamento das organizações, pois influencia diretamente tanto a conformidade com a legislação quanto a eficiência operacional. No Brasil, o artigo 74 da Consolidação das Leis do Trabalho (CLT)\footnote{\url{https://www.planalto.gov.br/ccivil_03/decreto-lei/del5452.htm}} determina que empresas com mais de 20 colaboradores devem manter o registro de ponto de seus funcionários \cite{brasil1943}. No entanto, micro e pequenas empresas (MPEs) enfrentam desafios significativos nesse aspecto, especialmente devido ao alto custo de sistemas eletrônicos de controle de ponto. Como alternativa, muitas recorrem a métodos manuais, considerados mais acessíveis inicialmente, mas que acabam aumentando os riscos de imprecisões, perdas de dados e até fraudes no acompanhamento das horas trabalhadas, comprometendo a eficácia na gestão do tempo \cite{miranda2023}.
A adoção de sistemas digitais de registro de ponto oferece benefícios importantes. Essas soluções eliminam a necessidade de equipamentos físicos específicos, permitindo o uso de dispositivos já disponíveis, como tablets, smartphones ou computadores. Dessa forma, tornam-se opções mais práticas e econômicas para a gestão eficiente do banco de horas \cite{FlorindoBianchi2022}.
 
De acordo com \textcite{gomes2023}, o registro de ponto digital tem se consolidado como uma solução eficaz para empresas que buscam otimizar a gestão de suas equipes sem comprometer o orçamento. Esse modelo proporciona maior precisão e transparência no acompanhamento das horas trabalhadas, além de reduzir os custos operacionais. Em contrapartida, a permanência no uso de métodos manuais pode levar a falhas significativas na supervisão da jornada de trabalho, prejudicando tanto empregadores quanto empregados. Um monitoramento inadequado impacta negativamente o cumprimento das obrigações legais e a transparência necessária para garantir a confiança mútua entre as partes \cite{abreu2016sistema}.

Este projeto tem como objetivo e desenvolver uma solução digital acessível e eficiente. Para isso, será desenvolvido um sistema de registro de ponto digital que utiliza tecnologias como escaneamento de QR Code e geolocalização. Essa abordagem busca otimizar o controle das horas trabalhadas, proporcionando maior precisão, autonomia aos funcionários e conformidade com as exigências legais, além de reduzir custos e riscos associados aos métodos manuais. 

Segundo \textcite{gomes2023}, a adoção de sistemas digitais para o registro de ponto oferece uma alternativa prática e econômica para empresas com recursos limitados, enquanto \textcite{mariotti2011} enfatizam que esses sistemas desempenham um papel essencial na melhoria dos processos internos e no cumprimento das normas trabalhistas. Assim, os sistemas digitais não apenas aumentam a eficiência, mas também promovem a transparência e a segurança jurídica para empregadores e funcionários.

\section{Justificativa}
Este projeto visa aprimorar o gerenciamento e controle das horas trabalhadas, beneficiando tanto empresas quanto funcionários, ao modernizar e tornar mais seguro o processo de gestão de pessoas por meio de soluções tecnológicas. Muitas micro e pequenas empresas (MPEs) ainda enfrentam dificuldades para substituir registros manuais por sistemas digitais. Métodos tradicionais, como planilhas e livros de ponto, são suscetíveis a erros humanos e não garantem a segurança necessária, podendo resultar em inconsistências no controle de jornada e problemas trabalhistas \cite{FlorindoBianchi2022}. 

Diante desse cenário, este estudo se torna importante para analisar e desenvolver uma solução digital acessível, segura e eficiente. Além de atender às exigências legais, um sistema digital pode reduzir custos operacionais, otimizar o controle de jornada e fortalecer a transparência nas relações de trabalho. Com isso, há potencial para impactos positivos significativos, como aumento da produtividade, redução de erros manuais e maior satisfação dos colaboradores \cite{gomes2023}.

De acordo com \textcite{Longo2019}, a transformação digital tem impulsionado mudanças rápidas e contínuas nas organizações, reformulando produtos, serviços e processos internos. Essas mudanças afetam diretamente as relações de trabalho e tornam indispensável a adoção de soluções inovadoras para acompanhar a evolução do mercado. Assim, este estudo se justifica pela necessidade de desenvolver uma ferramenta digital que atenda a essas novas demandas, promovendo benefícios para empregadores e funcionários, além de contribuir para a modernização e eficiência da gestão empresarial.


\section{Objetivos}

\subsection{Geral}

Desenvolver um MVP (Minimum Viable Product) de sistema web de Registro de Ponto com geolocalização para pequenas e médias empresas, permitindo que funcionários registrem suas entradas e saídas através de smartphones com validação de localização, oferecendo dashboard gerencial para acompanhamento de bancos de horas, gestão de justificativas e relatórios automatizados, visando proporcionar maior controle, transparência e eficiência no controle de ponto.


\subsection{Específicos}
\begin{itemize}
    \item Analisar as principais ferramentas digitais de registro de ponto disponíveis no mercado, identificando suas funcionalidades, limitações e requisitos para garantir um controle eficiente da jornada de trabalho.
    \item Identificar as principais dificuldades enfrentadas por empresas e funcionários no controle de ponto.
    \item Desenvolver a arquitetura do MVP utilizando Next.js e NestJS, implementando o sistema de autenticação JWT e os perfis de acesso (funcionário e gerente).
    \item Implementar o módulo de registro de ponto para o funcionário, com validação de geolocalização por raio configurável via smartphone.
    \item Construir o painel gerencial, englobando a gestão de funcionários, o fluxo de aprovação de justificativas, o cálculo automático de banco de horas e a emissão de relatórios.
\end{itemize}

\section{Metodologia}

A metodologia adotada para este projeto combinou uma pesquisa exploratória e qualitativa com uma abordagem ágil de engenharia de software, estruturando o trabalho em duas frentes principais e complementares: a primeira, focada no embasamento teórico e na análise de mercado para a fundamentação do projeto, e a segunda, na construção prática e iterativa do Mínimo Produto Viável (MVP).

% \subsection{Classificação da Pesquisa}

% O presente trabalho caracteriza-se como uma pesquisa de natureza \textbf{aplicada}, pois visa à criação de um produto tecnológico — um Mínimo Produto Viável (MVP) — para a solução de um problema prático de mercado: o controle de ponto para pequenas e médias empresas. Do ponto de vista dos objetivos, a pesquisa é \textbf{exploratória}, uma vez que investigou o cenário de ferramentas concorrentes para identificar lacunas, e \textbf{descritiva}, ao documentar detalhadamente todo o processo de desenvolvimento e a arquitetura do software. Por fim, quanto à abordagem, o estudo é predominantemente \textbf{qualitativo}, com foco na análise das funcionalidades implementadas, na qualidade da arquitetura de software e na avaliação heurística da interface, sem a pretensão de coletar dados estatísticos de uso.

\subsection{Procedimentos de Pesquisa Exploratória}

\subsubsection{Pesquisa Bibliográfica} 

A pesquisa bibliográfica foi fundamental para compreender os conceitos fundamentais de controle de ponto, legislação trabalhista brasileira e tecnologias de geolocalização. Foram analisados artigos científicos, livros técnicos e documentações oficiais sobre sistemas de registro de ponto, com foco especial na Consolidação das Leis do Trabalho (CLT) e suas exigências para empresas com mais de 20 funcionários. A pesquisa também abrangeu estudos sobre usabilidade em sistemas móveis e boas práticas de desenvolvimento web.

\subsubsection{Análise de Mercado e Soluções Concorrentes}

Foi realizada uma análise exploratória das principais ferramentas digitais de registro de ponto disponíveis no mercado brasileiro, identificando suas funcionalidades, limitações e requisitos para garantir um controle eficiente da jornada de trabalho. Esta pesquisa permitiu identificar lacunas no mercado, especialmente para pequenas e médias empresas que não possuem recursos para sistemas complexos e caros. A análise incluiu sistemas como PontoTel, Tangerino, PontoMais e outros, mapeando suas características técnicas, custos e adequação para diferentes portes de empresa.

\subsection{Metodologia de Desenvolvimento do Software (MVP)}

\subsubsection{Modelo de Processo de Desenvolvimento}

O desenvolvimento do MVP seguiu uma abordagem ágil com prototipagem evolutiva, utilizando a metodologia Frontend-First — uma estratégia que prioriza a construção da experiência do usuário, desenvolvendo inicialmente todas as interfaces com dados estáticos (mockados) utilizando Next.js com TypeScript e componentes Radix UI. O backend foi desenvolvido posteriormente com NestJS, construindo a API de forma direcionada para atender às necessidades já estabelecidas no frontend.

A arquitetura foi desenvolvida seguindo princípios de separação de responsabilidades, com frontend e backend como serviços independentes, utilizando PostgreSQL como banco de dados principal e implementando autenticação JWT para controle de acesso diferenciado entre funcionários e gestores.

\subsubsection{Verificação Técnica e Testes} 

Para a verificação do MVP, considerando o estágio de desenvolvimento, foi implementada uma abordagem de verificação interna em duas frentes. A primeira consistiu na Verificação de Requisitos Funcionais, onde cada funcionalidade especificada foi testada sistematicamente para garantir seu correto funcionamento. Os principais módulos verificados foram:
\begin{itemize}
    \item Registro de ponto com validação por geolocalização;
    \item Gestão de funcionários, perfis e departamentos;
    \item Fluxo de envio e aprovação de justificativas;
    \item Cálculo automático e exibição do banco de horas.
\end{itemize}

A segunda frente foi uma Avaliação Heurística da interface do usuário, baseada nas 10 Heurísticas de Usabilidade de Jakob Nielsen. Esta análise focou nos principais fluxos de interação, com atenção especial à usabilidade em dispositivos móveis, considerando que o registro de ponto é realizado via smartphone. A avaliação identificou potenciais problemas de usabilidade e garantiu que o MVP adere a boas práticas de design, como consistência visual, feedback ao usuário e prevenção de erros.

