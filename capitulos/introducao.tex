% ----------------------------------------------------------
% Introdução
% ----------------------------------------------------------
\chapter{Introdução}

A gestão da jornada de trabalho é um elemento crucial para o bom funcionamento das organizações, pois influencia diretamente tanto a conformidade com a legislação quanto a eficiência operacional. No Brasil, o artigo 74 da Consolidação das Leis do Trabalho (CLT)\footnote{\url{https://www.planalto.gov.br/ccivil_03/decreto-lei/del5452.htm}} determina que empresas com mais de 20 colaboradores devem manter o registro de ponto de seus funcionários \cite{brasil1943}. No entanto, micro e pequenas empresas (MPEs) enfrentam desafios significativos nesse aspecto, especialmente devido ao alto custo de sistemas eletrônicos de controle de ponto. Como alternativa, muitas recorrem a métodos manuais, considerados mais acessíveis inicialmente, mas que acabam aumentando os riscos de imprecisões, perdas de dados e até fraudes no acompanhamento das horas trabalhadas, comprometendo a eficácia na gestão do tempo \cite{miranda2023}.

A adoção de sistemas digitais de registro de ponto oferece benefícios importantes. Essas soluções eliminam a necessidade de equipamentos físicos específicos, permitindo o uso de dispositivos já disponíveis, como tablets, smartphones ou computadores. Dessa forma, tornam-se opções mais práticas e econômicas para a gestão eficiente do banco de horas \cite{FlorindoBianchi2022}.
 
De acordo com \textcite{gomes2023}, o registro de ponto digital tem se consolidado como uma solução eficaz para empresas que buscam otimizar a gestão de suas equipes sem comprometer o orçamento. Esse modelo proporciona maior precisão e transparência no acompanhamento das horas trabalhadas, além de reduzir os custos operacionais. Em contrapartida, a permanência no uso de métodos manuais pode levar a falhas significativas na supervisão da jornada de trabalho, prejudicando tanto empregadores quanto empregados. Um monitoramento inadequado impacta negativamente o cumprimento das obrigações legais e a transparência necessária para garantir a confiança mútua entre as partes \cite{abreu2016sistema}.

Este projeto tem como objetivo e desenvolver uma solução digital acessível e eficiente. Para isso, será desenvolvido um sistema de registro de ponto digital que utiliza tecnologias como escaneamento de QR Code e geolocalização. Essa abordagem busca otimizar o controle das horas trabalhadas, proporcionando maior precisão, autonomia aos funcionários e conformidade com as exigências legais, além de reduzir custos e riscos associados aos métodos manuais. 

Segundo \textcite{gomes2023}, a adoção de sistemas digitais para o registro de ponto oferece uma alternativa prática e econômica para empresas com recursos limitados, enquanto \textcite{mariotti2011} enfatizam que esses sistemas desempenham um papel essencial na melhoria dos processos internos e no cumprimento das normas trabalhistas. Assim, os sistemas digitais não apenas aumentam a eficiência, mas também promovem a transparência e a segurança jurídica para empregadores e funcionários.

\section{Justificativa}
Este projeto visa aprimorar o gerenciamento e controle das horas trabalhadas, beneficiando tanto empresas quanto funcionários, ao modernizar e tornar mais seguro o processo de gestão de pessoas por meio de soluções tecnológicas. Muitas micro e pequenas empresas (MPEs) ainda enfrentam dificuldades para substituir registros manuais por sistemas digitais. Métodos tradicionais, como planilhas e livros de ponto, são suscetíveis a erros humanos e não garantem a segurança necessária, podendo resultar em inconsistências no controle de jornada e problemas trabalhistas \cite{FlorindoBianchi2022}. 

Diante desse cenário, este estudo se torna importante para analisar e desenvolver uma solução digital acessível, segura e eficiente. Além de atender às exigências legais, um sistema digital pode reduzir custos operacionais, otimizar o controle de jornada e fortalecer a transparência nas relações de trabalho. Com isso, há potencial para impactos positivos significativos, como aumento da produtividade, redução de erros manuais e maior satisfação dos colaboradores \cite{gomes2023}.

De acordo com \textcite{Longo2019}, a transformação digital tem impulsionado mudanças rápidas e contínuas nas organizações, reformulando produtos, serviços e processos internos. Essas mudanças afetam diretamente as relações de trabalho e tornam indispensável a adoção de soluções inovadoras para acompanhar a evolução do mercado. Assim, este estudo se justifica pela necessidade de desenvolver uma ferramenta digital que atenda a essas novas demandas, promovendo benefícios para empregadores e funcionários, além de contribuir para a modernização e eficiência da gestão empresarial.


\section{Objetivos}

\subsection{Geral}

Desenvolver um MVP (Minimum Viable Product) de sistema web de Registro de Ponto com geolocalização para pequenas e médias empresas, permitindo que funcionários registrem suas entradas e saídas através de smartphones com validação de localização, oferecendo dashboard gerencial para acompanhamento de bancos de horas, gestão de justificativas e relatórios automatizados, visando proporcionar maior controle, transparência e eficiência no controle de ponto.

\subsection{Específicos}
\begin{itemize}
    \item Analisar as principais ferramentas digitais de registro de ponto disponíveis no mercado, identificando suas funcionalidades, limitações e requisitos para garantir um controle eficiente da jornada de trabalho.
    \item Identificar as principais dificuldades enfrentadas por empresas e funcionários no controle de ponto.
    \item Desenvolver a arquitetura do MVP utilizando Next.js e NestJS, implementando o sistema de autenticação JWT e os perfis de acesso (funcionário e gerente).
    \item Implementar o módulo de registro de ponto para o funcionário, com validação de geolocalização por raio configurável via smartphone.
    \item Construir o painel gerencial, englobando a gestão de funcionários, o fluxo de aprovação de justificativas, o cálculo automático de banco de horas e a emissão de relatórios.
\end{itemize}

\section{Metodologia}
A metodologia empregada neste projeto foi estruturada em três etapas sequenciais e interdependentes: pesquisa exploratória para fundamentação do problema, desenvolvimento ágil do Mínimo Produto Viável (MVP) e, por fim, a avaliação técnica da solução.

A primeira etapa, de pesquisa exploratória, teve como objetivo aprofundar o entendimento sobre os desafios do controle de jornada de trabalho para micro e pequenas empresas. Para isso, foi realizada uma pesquisa bibliográfica sobre a legislação trabalhista pertinente, tecnologias de geolocalização e boas práticas de desenvolvimento de software. Em paralelo, conduziu-se uma pesquisa de mercado qualitativa com uma abordagem dupla: foi aplicado um questionário direcionado a gestores, para mapear os desafios operacionais e administrativos (detalhado no APÊNDICE A), e outro questionário específico para funcionários, a fim de compreender suas percepções e frustrações com os sistemas atuais (detalhado no APÊNDICE B). Os achados desta pesquisa de campo, que apontaram para a necessidade de maior transparência e automação, foram cruciais para a especificação dos requisitos do sistema.

A segunda etapa, de desenvolvimento do software, seguiu um processo iterativo alinhado a práticas de prototipagem evolutiva, onde o software é construído e refinado em ciclos \cite{sommerville2011software}. Adotou-se a estratégia \textit{Frontend-First}, que prioriza a construção da interface do usuário (UI) como guia para o desenvolvimento do sistema. Utilizando Next.js com TypeScript e componentes Radix UI, as interfaces foram inicialmente desenvolvidas com dados estáticos (\textit{mockados}), permitindo a validação dos fluxos de interação antes da implementação da lógica de negócio no \textit{backend}, que foi construído de forma direcionada com NestJS.

Por fim, a terceira etapa consistiu na avaliação do MVP. Considerando a impossibilidade de realizar testes com usuários finais, optou-se por uma verificação interna em duas frentes. A primeira foi uma \textit{Verificação de Requisitos Funcionais}, na qual se analisou sistematicamente o atendimento aos requisitos especificados. A segunda foi uma \textit{Avaliação Heurística}, um método de inspeção consolidado para encontrar problemas de usabilidade em interfaces \cite{Nielsen1994}. Atuando como avaliadora especialista, foram inspecionados os principais fluxos do sistema, com foco na experiência em dispositivos móveis, com base nas 10 heurísticas de usabilidade de Nielsen, cujos resultados são apresentados no capítulo de Resultados e Discussão.